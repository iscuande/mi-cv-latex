\documentclass[11pt]{article}
\usepackage[utf8]{inputenc}
\usepackage[T1]{fontenc}
\usepackage[spanish]{babel}
\usepackage[margin=1.5cm]{geometry}
\usepackage{enumitem}
\setlength{\parskip}{0.4em}
\setlist{noitemsep, topsep=0.2em}

\begin{document}

% ===== ENCABEZADO =====
\noindent{\Huge\textbf{Jorge Armando Cordero García}}\\[0.1em]
\large\textbf{Ingeniero de Sistemas | Gestión TIC Pública y Privada}\\[0.2em]
\noindent Colombia | +57 317 819 5354 | \texttt{iscuande\_1982@hotmail.com} | linkedin.com/in/jacoga
\vspace{0.3em}\hrule\vspace{0.5em}

% ===== PERFIL =====
\section*{Perfil Profesional}
Ingeniero de Sistemas con más de 10 años de experiencia en gestión tecnológica integral en sectores público y privado. Especializado en administración de infraestructura TIC, soporte técnico multinivel, gestión de proyectos tecnológicos y transformación digital institucional. Formación complementaria en liderazgo público, derechos humanos y ética en la función pública. Certificado en herramientas tecnológicas de vanguardia (Google, Cisco) con enfoque en innovación y eficiencia operativa.

% ===== EXPERIENCIA =====
\section*{Experiencia Profesional}

\noindent\textbf{Ingeniero de Sistemas - Consultor Independiente} \hfill \textbf{Jun 2024 -- Presente}\\
\textit{Sectores: Salud, Educación, Gobierno Local, PYMES} \hfill \textit{Multimodal}
\begin{itemize}
    \item Asesoría tecnológica integral para optimización de procesos institucionales.
    \item Implementación de soluciones TIC para mejora de servicios ciudadanos.
    \item Capacitación y transferencia de conocimiento a funcionarios públicos.
    \item Desarrollo de estrategias de transformación digital adaptadas a contextos locales.
\end{itemize}

\noindent\textbf{Ingeniero de Sistemas} \hfill \textbf{Feb 2023 -- Jul 2024}\\
\textit{E.S.E. Hospital Olaya Herrera - Gamarra, Cesar} \hfill \textit{Presencial}
\begin{itemize}
    \item Administración del sistema de información asistencial en salud pública.
    \item Gestión de infraestructura tecnológica hospitalaria (servidores, redes, equipos).
    \item Soporte técnico a 200+ usuarios en entorno de misión crítica.
    \item Garantía de continuidad operativa del servicio de salud.
\end{itemize}

\noindent\textbf{Ingeniero de Soporte y Gestión TIC} \hfill \textbf{Feb 2022 -- Jul 2022}\\
\textit{Universidad Popular del César - Aguachica, Cesar} \hfill \textit{Presencial}
\begin{itemize}
    \item Administración de infraestructura tecnológica educativa.
    \item Liderazgo en proyectos de incorporación TIC en procesos académicos.
    \item Control de inventario y gestión de activos tecnológicos institucionales.
    \item Capacitación docente en herramientas digitales educativas.
\end{itemize}

\noindent\textbf{Ingeniero de Sistemas} \hfill \textbf{Ene 2017 -- Dic 2017}\\
\textit{Cordisco - Bogotá, D.C.} \hfill \textit{Presencial}
\begin{itemize}
    \item Administración de red corporativa y sistemas de información empresarial.
    \item Implementación de mejoras tecnológicas para desarrollo organizacional.
    \item Gestión de proyectos de fortalecimiento institucional.
    \item Asesoría en adopción de nuevas tecnologías.
\end{itemize}

\noindent\textbf{Líder GELT y Gestión TIC} \hfill \textbf{Jun 2015 -- Feb 2016}\\
\textit{Alcaldía Municipal de Gamarra, Cesar} \hfill \textit{Presencial}
\begin{itemize}
    \item Liderazgo de la Gestión Electrónica (GELT) municipal.
    \item Administración de servidores y red LAN institucional.
    \item Gestión de recursos TIC y procesos de contratación pública.
    \item Articulación interinstitucional para proyectos tecnológicos locales.
\end{itemize}

% ===== FORMACIÓN ACADÉMICA =====
\section*{Formación Académica}
\begin{itemize}[label={}]
    \item \textbf{Ingeniería de Sistemas} \hfill \textbf{2015 -- 2023}\\
    Universidad Popular del César - Aguachica, Cesar
    
    \item \textbf{Especialización en TIC para Diseño de Estrategias Didácticas} \hfill \textbf{2022 -- 2023}\\
    Universidad Popular del César - En proceso de grado
\end{itemize}

% ===== CERTIFICACIONES PÚBLICAS Y LIDERAZGO =====
\section*{Certificaciones en Gestión Pública y Liderazgo}
\begin{itemize}
    \item \textbf{Liderazgo en la Función Pública} - Escuela Superior de Administración Pública (ESAP)
    \item \textbf{Derechos Humanos y Construcción de Paz} - Programa de las Naciones Unidas para el Desarrollo (PNUD)
    \item \textbf{Ética Pública y Transparencia} - Contraloría General de la República
    \item \textbf{Gestión de Contratos Estatales} - Colombia Compra Eficiente
    \item \textbf{Planificación y Gerencia de Proyectos} - SENA Virtual
\end{itemize}

% ===== CERTIFICACIONES TECNOLÓGICAS PROFESIONALES =====
\section*{Certificaciones Tecnológicas}
\begin{itemize}
    \item \textbf{Google Professional Certificates:}\\
    - IT Support Specialist (2023)\\
    - Data Analytics (2023)\\
    - Cybersecurity (En curso - 2025)
    
    \item \textbf{Cisco Networking Academy:}\\
    - IT Customer Support Basics (2023)\\
    - Introduction to IoT (2023)\\
    - Introduction to Cybersecurity (2023)
    
    \item \textbf{Inteligencia Artificial Aplicada:}\\
    - Fundamentos de IA para Negocios - Google (2024)\\
    - Herramientas de IA Generativa - Microsoft (2024)
\end{itemize}

% ===== HABILIDADES TÉCNICAS =====
\section*{Competencias Técnicas}
\begin{itemize}[label={}]
    \item \textbf{Gestión TIC Pública:} Administración de sistemas institucionales, contratación pública tecnológica, gobierno electrónico, planes de tecnología.
    \item \textbf{Infraestructura Tecnológica:} Redes LAN/WAN, servidores locales, virtualización, cableado estructurado, soporte multinivel.
    \item \textbf{Desarrollo Institucional:} Transformación digital, capacitación tecnológica, gestión del cambio, proyectos TIC.
    \item \textbf{Herramientas:} Microsoft Office Suite, Google Workspace, herramientas de gestión pública (SUIT, SIGEP, etc.)
\end{itemize}

% ===== HABILIDADES TRANSVERSALES =====
\section*{Competencias Transversales}
Liderazgo institucional, comunicación asertiva, trabajo en equipo interdisciplinario, resolución de conflictos, pensamiento estratégico, adaptabilidad, gestión bajo presión, orientación al servicio ciudadano, ética profesional, transparencia.

% ===== IDIOMAS =====
\section*{Idiomas}
\begin{itemize}[label={}]
    \item \textbf{Español:} Nativo
    \item \textbf{Inglés:} Nivel básico-técnico (lectura y escritura de documentación técnica)
\end{itemize}

% ===== DISPONIBILIDAD =====
\section*{Disponibilidad y Modalidad}
\begin{itemize}[label={}]
    \item \textbf{Disponibilidad:} Inmediata
    \item \textbf{Modalidad:} Presencial, Remota o Híbrida según requerimiento institucional
    \item \textbf{Movilidad:} Disponible para ubicación en cualquier municipio/departamento
\end{itemize}

\end{document}