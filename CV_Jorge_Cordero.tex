\documentclass[11pt, a4paper]{article}
\usepackage[utf8]{inputenc}
\usepackage[T1]{fontenc}
\usepackage[spanish]{babel}
\usepackage[left=1.5cm, right=1.5cm, top=1.5cm, bottom=1.5cm]{geometry}
\usepackage{enumitem}
\usepackage{hyperref}
\usepackage{array}
\usepackage{parskip}
\usepackage{titlesec}
\usepackage{fontawesome5}
\usepackage{xcolor}
\usepackage{setspace}

% ===== COLORES Y FORMATOS =====
\definecolor{blueheader}{RGB}{0, 70, 130}
\definecolor{graytext}{RGB}{80, 80, 80}

\hypersetup{
    colorlinks=true,
    urlcolor=blueheader,
    linkcolor=blueheader,
}

% ===== FORMATO DE SECCIONES =====
\titleformat{\section}
{\Large\bfseries\color{blueheader}}
{}
{0em}
{\rule{\linewidth}{1pt}\\[0.2em]}
[\vspace{-0.5em}\rule{\linewidth}{0.5pt}]

\titlespacing*{\section}{0pt}{0.8em}{0.4em}

% ===== ENCABEZADO =====
\newcommand{\header}[5]{
    \begin{center}
        {\Huge\bfseries #1}\\[0.2em]
        {\large\bfseries #2}\\[0.5em]
        \faMapMarker\ \textbf{#3} \hspace{1em}
        \faPhone\ \textbf{#4} \hspace{1em}
        \faEnvelope\ \href{mailto:#5}{\underline{#5}}\\[0.3em]
        \faLinkedin\ \href{https://linkedin.com/in/jacoga}{linkedin.com/in/jacoga}
    \end{center}
    \vspace{0.5em}
}

% ===== EXPERIENCIA =====
\newenvironment{experience}
{\begin{itemize}[leftmargin=*, label={}, itemsep=0.5em]}
{\end{itemize}}

\newcommand{\expitem}[4]{
    \item
    \textbf{#1} \hfill \textbf{#2}\\
    \textit{#3} \hfill #4
}

% ===== HABILIDADES =====
\newcommand{\skillcategory}[2]{
    \textbf{#1:} #2
}

\begin{document}

% ===== ENCABEZADO PRINCIPAL =====
\header
{Jorge Armando Cordero García}
{Ingeniero de Sistemas | Especialista en Soporte TI e Infraestructura}
{Gamarra, Colombia}
{+57 317 819 5354}
{iscuande_1982@hotmail.com}

% ===== PERFIL PROFESIONAL =====
\section*{Perfil Profesional}
Ingeniero de Sistemas con más de 10 años de experiencia en \textbf{Soporte TI Nivel 2}, \textbf{infraestructura tecnológica} y \textbf{redes LAN/WAN} en sectores críticos como salud, educación y gobierno. Especializado en gestión de incidencias (\textbf{ITIL}), continuidad operativa y soporte remoto (\textbf{Help Desk}). En formación continua en \textbf{ciberseguridad}, orientando mi crecimiento profesional hacia roles de infraestructura segura y seguridad de la información.

% ===== EXPERIENCIA PROFESIONAL =====
\section*{Experiencia Profesional}
\begin{experience}

\expitem{Especialista TI - Consultor Externo}{Ene 2024 -- Presente}
{Tictelco S.A.S / Expansiones Digitales / Orkasun Roof (USA) / Clientes Privados}{Modalidad Híbrida (Remota/Presencial)}
\begin{itemize}[leftmargin=*, label=\textbullet]
    \item Soporte técnico especializado bajo demanda para múltiples organizaciones.
    \item Resolución de incidencias de infraestructura, diagnóstico de conectividad en sistemas Windows.
    \item Adecuación y mantenimiento de racks para equipos de switching y comunicaciones.
    \item Soporte externo en mantenimiento de plataformas WordPress, gestión de contenidos y ajustes técnicos en HTML/CSS.
    \item Soporte de escritorio independiente, optimización de sistemas operativos y capacitación técnica.
\end{itemize}

\expitem{Ingeniero de Sistemas}{Feb 2023 -- Jul 2024}
{E.S.E. Hospital Olaya Herrera}{Presencial}
\begin{itemize}[leftmargin=*, label=\textbullet]
    \item Soporte TI en entorno hospitalario crítico para más de 200 usuarios.
    \item Administración básica de servidores Windows Server y gestión de redes LAN.
    \item Resolución de incidencias garantizando continuidad operativa del 99.5\%.
\end{itemize}

\expitem{Ingeniero de Sistemas}{Feb 2022 -- Jul 2022}
{Universidad Popular del César}{Presencial}
\begin{itemize}[leftmargin=*, label=\textbullet]
    \item Soporte a infraestructura tecnológica institucional.
    \item Mantenimiento de redes y control de inventario TI.
\end{itemize}

\expitem{Ingeniero de Sistemas}{Abr 2017 -- Oct 2017}
{Cordisco}{Presencial}
\begin{itemize}[leftmargin=*, label=\textbullet]
    \item Administración de red y soporte TI a usuarios internos.
    \item Implementación de mejoras tecnológicas.
\end{itemize}

\expitem{Ingeniero de Sistemas}{Abr 2016 -- Mar 2017}
{Linktech de Colombia}{Presencial}
\begin{itemize}[leftmargin=*, label=\textbullet]
    \item Soporte a sistemas de información y administración básica de redes.
\end{itemize}

\end{experience}

% ===== EXPERIENCIA ADICIONAL =====
\section*{Experiencia Adicional}
\begin{experience}
\expitem{Auxiliar de Inspección}{Nov 2017 -- Feb 2018}
{Bureau Veritas Colombia Ltda}{Presencial}
\begin{itemize}[leftmargin=*, label=\textbullet]
    \item Inspección de medidores y elaboración de informes técnicos en Excel.
\end{itemize}

\expitem{Técnico Operativo y Administrativo de Apoyo en Tecnología}{Feb 2013 -- Feb 2016}
{Sector Gobierno}{Presencial}
\begin{itemize}[leftmargin=*, label=\textbullet]
    \item Apoyo a gestión TIC, Gobierno en Línea y soporte tecnológico básico.
\end{itemize}

\expitem{Soldado Profesional}{Nov 2000 -- Jun 2005}
{Ejército Nacional de Colombia}{Presencial}
\begin{itemize}[leftmargin=*, label=\textbullet]
    \item Trabajo en entornos de alta presión, liderazgo operativo y disciplina institucional.
\end{itemize}
\end{experience}

% ===== CERTIFICACIONES =====
\section*{Certificaciones y Formación Continua}
\begin{itemize}[leftmargin=*, label=\textbullet]
    \item \textbf{Google IT Support} (2023)
    \item \textbf{Google Data Analytics} (2023)
    \item \textbf{Google Cybersecurity} (En curso, estimación: 2025)
    \item \textbf{Cisco Networking Academy: IT Customer Support Basics} (2023)
    \item \textbf{Cisco Networking Academy: Introduction to IoT} (2023)
\end{itemize}

% ===== HABILIDADES TÉCNICAS =====
\section*{Habilidades Técnicas}
\begin{itemize}[leftmargin=*, label={}]
    \item \skillcategory{Soporte TI}{Nivel 1 y 2, Help Desk, Gestión de Incidencias (ITIL), Soporte Remoto (TeamViewer, AnyDesk)}
    \item \skillcategory{Redes}{LAN/WAN, TCP/IP, Switching, Rack Management, Conectividad}
    \item \skillcategory{Sistemas Operativos}{Windows Server (básico), Linux (básico), Active Directory (básico)}
    \item \skillcategory{Ciberseguridad}{Fundamentos, Hardening, Awareness}
    \item \skillcategory{Desarrollo Web}{HTML5, CSS3, JavaScript (básico), WordPress, Diseño Responsivo}
\end{itemize}

% ===== HABILIDADES TRANSVERSALES =====
\section*{Habilidades Transversales}
Resolución de problemas, comunicación asertiva, atención al detalle, pensamiento crítico, adaptabilidad, trabajo bajo presión, orientación al usuario, liderazgo operativo.

% ===== IDIOMAS =====
\section*{Idiomas}
\begin{itemize}[leftmargin=*, label={}]
    \item \textbf{Español:} Nativo
    \item \textbf{Inglés:} Básico (A2 – Comprensión y escritura técnica)
\end{itemize}

% ===== NOTA ATS =====
\vspace{1em}
{\footnotesize \color{graytext} \textbf{Nota:} Este CV está optimizado para sistemas ATS (Applicant Tracking Systems) con palabras clave relevantes y estructura clara.}

\end{document}
